\section{Objet Métier Client}

\begin{description}
\item{SOMC1 Consultation Client}
\item{SOMC2 Consultation Liste des Client}

\item{SBC1 Consultation Entité Personne}
\item{SBC1 Consultation Entité Compte}
\end{description}

\subsection{SM1 Liste des comptes d'un Client}
\paragraph{SOM} SOMC1 : Consultation Client - SOMP1 : Consultation produit
\paragraph{Entrées} id client
\paragraph{Sorties} NomClient, Adr. Client
\paragraph{Description}
Consultation de l'ensemble des informations client et récupération du type de compte à partir de sa fiche produit

\begin{tabular}{|c|c|}
\hline
Stimulis 
& SM \\
\hline
Cliquer sur le bouton "Contact" de la page d'accueil de la gestion des contacts commerciaux puis choisir un état dans la liste "état" puis appuyer sur le bouton "Rechercher"
& SM1 (ConsulterContactsPrevus) : Affichage par agence de tous les contacts prévus (pour Chef agence)\\
\hline
Saisir le nom d'un agent puis cliquer sur le bouton "Rechercher"
& SM2 (ConsulterContactsAgent) : Consulter les contacts de l'agent sélectionné (pour Chef agence)\\
\hline
Cliquer sur le bouton "Sélectionner Agent"
& SM3 (ConsulterAgents) : Consulter les agents de l'agence\\
\hline
Cliquer sur le bouton "Sélectionner Client" puis choisir un client
& SM4 (ConsulterClients) : Consulter les clients de l'agence\\
\hline
Sélectionner un contact et cliquer sur le bouton "Réaffecter", il faudra ensuite sélectionner un agent puis valider ce choix
& SM5 (AffecterContactPrevu) : Affecter un contact prévu un agent (pour chef d'agence)\\
\hline
Sélectionner un contact et cliquer sur le bouton "Annuler", saisir un motif d'annulation (raison) puis valider
& SM6 (AnnulerContactPrevu) : Annuler un contact\\
\hline
Sélectionner des contacts e les cochant et cliquer sur le bouton "Regrouper"
& SM7 (RegrouperContacts) : Regrouper des contacts\\
\hline
Appui sur le bouton "Entretien"
& SM8 (PréparerContact) : Préparation et suivi entretien\\
\hline
\end{tabular}


\section{Spécification de la couche noyau du bloc applicatif Contact}

       \subsection{SM1 : ConsulterContactsPrevus}

       \subsection{SM2 : ConsulterContactsAgent}

       \subsection{SM3 : AffecterContactPrevu}

       \subsection{SM4 : AnnulerContactPrevu}

       \subsection{SM5 : RegrouperContacts}