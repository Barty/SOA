\begin{tabular}{|c|c|}
\hline
Stimulis 
& SM \\
\hline
Cliquer sur le bouton "Contact" de la page d'accueil de la gestion des contacts commerciaux puis choisir un �tat dans la liste "�tat" puis appuyer sur le bouton "Rechercher"
& SM1 : Affichage par agence de tous les contacts pr�vus (pour Chef agence)\\
\hline
Saisir le nom d'un agent puis cliquer sur le bouton "Rechercher"
& SM2 : Consulter les contacts de l'agent s�lectionn� (pour Chef agence)\\
\hline
Cliquer sur le bouton "S�lectionner Agent"
& SM3 : Consulter les agents de l'agence\\
\hline
Cliquer sur le bouton "S�lectionner Client" puis choisir un client
& SM4 : Consulter les clients de l'agence\\
\hline
S�lectionner un contact et cliquer sur le bouton "R�affecter", il faudra ensuite s�lectionner un agent puis valider ce choix
& SM5 : Affecter un contact pr�vu � un agent (pour chef d'agence)\\
\hline
S�lectionner un contact et cliquer sur le bouton "Annuler", saisir un motif d'annulation (raison) puis valider
& SM6 : Annuler un contact\\
\hline
S�lectionner des contacts en les cochant et cliquer sur le bouton "Regrouper"
& SM7 : Regrouper des contacts\\
\hline
Appui sur le bouton "Entretien"
& SM8 (Pr�parerContact) : Pr�paration et suivi entretien\\
\hline
Appui sur le bouton "D�tails" d'un Rendez-vous, ou double cliquer sur la plage horaire associ� au Rendez-vous
& SM9 : Affichage de la fen�tre des d�tails d'un Rendez-vous\\
\hline
Appui sur le bouton "Modifier" dans la f�netre d�tails d'un Rendez-vous
& SM9 : Passage en mode �dition du Rendez-vous\\
\hline
Appui sur le bouton "Supprimer" dans la f�netre d�tails d'un Rendez-vous, ou glisser la plage horaire associ� en dehors de la fen�tre
& SM9 : Suppr�ssion du Rendez-vous\\
\hline
Appui sur le bouton "R�affecter" dans la f�netre d�tails d'un Rendez-vous
& SM9 : R�affectation d'une Rendez-vous\\
\hline
Clic sur une plage horaire libre.
& SM9 : Cr�ation d'un rendez-vous\\
\hline
\end{tabular}


\section{Spécification de la couche noyau du bloc applicatif Contact}

\subsection{Service Métiers}
\subsubsection{SM1: ConsulterContactsPrevus (pour chef agence)}
\begin{itemize}
\item E : idClient
\item S : liste $<$idContactPrevu, motif, nomClient, nomAgent, datesRDV$>$
\item SOM utilisés: 	
\begin{itemize}		
\item SOM1 : ConsulterClients (Bloc Client)
\item SOM2 : ConsulterContactsPrevus (Bloc Contact)
\item SOM3 : AgentAffecter (Bloc Structure)
\item SOM4 : ConsulterInfosClient (Bloc Client)	
\end{itemize}		
\item Procédure : 	
\begin{itemize}		
\item Recherche des clients de l’agence. (SOM1)
\item Pour chaque client
\begin{itemize}
\item Recherche la liste des contacts prévus et les motifs (SOM2)
\item Recherche de l’agent affecté. (SOM3)
\item Recherche du nom (SOM4)
\end{itemize}
\end{itemize}
\end{itemize}
\subsubsection{SM2 : ConsulterContactsAgent}
\begin{itemize}
\item E : idAgent, idClient, etatContact 
\item S : liste  $<$idContactPrevu, motif, nomClient, nomAgent, datesRDV$>$
\item SOM utilisés: 	
\begin{itemize}		
\item SOM2 : ConsulterContactsPrévus (Bloc Contact)
\item SOM4 : ConsulterInfosClient (Bloc Client)
\item SOM5 : ConsulterDatesRDV (Bloc Agenda)
\end{itemize}		
\item Procédure : 	
\begin{itemize}		
\item Recherche les contacts prévus (avec motifs, dates RDV) pour un agent donné (SOM2)
\item Pour chaque contact
\begin{itemize}
\item Recherche le nom du client (SOM4)
\item Recherche les dates de RDV (SOM5)
\end{itemize}.
\end{itemize}.
\end{itemize}
\subsubsection{SM3 : AffecterContactPrevu}
\begin{itemize}
\item E : idContact, idAgent 	
\item S : Aucune (L’état du contact passe de prévu à affecté)
\item SOM utilisés: 	
\begin{itemize}		
\item SOM6 : AffecterContactPrevu	
\end{itemize}		
\item Procédure : 	
\begin{itemize}		
\item Associe à un agent un contact prévu (SOM6)
\end{itemize}.
\end{itemize}
\subsubsection{SM 4: AnnulerContactPrévu}
\begin{itemize}
\item E : idContact, raison annulation
\item S : validation d'action
\item SOM utilisés: 	
\begin{itemize}		
\item SOM7: AnnulerContactPrevu (idContact,Raison) (bloc Contact)	
\end{itemize}		
\item Procédure : 	
\begin{itemize}		
\item Changer l’état du contact ayant comme identification : idContact à l’état Annulé en utilisant SB MajEtat
\end{itemize}.
\end{itemize}
\subsubsection{SM5 : RegrouperContacts}
\begin{itemize}
\item E : list $<$listeIdContact$>$
\item S : validation d'action
\item SOM utilisés: 	
\begin{itemize}		
\item SOM8: RegrouperContacts(list $<$listeIdContact$>$) [Bloc Contact]		
\end{itemize}	
\item Procédure : 	
\begin{itemize}		
\item Regrouper tous les contacts dont id dans la liste passée
\end{itemize}.
\end{itemize}

\subsubsection{SM6: GenererContact}
\begin{itemize}
\item E : aucun
\item S : aucun
\item SOM utilisés: 	
\begin{itemize}		
\item SOM9 : recupererDerniersEvenements() (Bloc Evenement)
\item SOM10 : recupererClientsLies (Evenement) (Bloc Evenement)
\item SOM11 : creerContactPrevu (Bloc Client)	
\end{itemize}		
\item Procédure : 	
\begin{itemize}		
\item Récuperer tous les événéments récemment captés.(SOM9)
\item Pour chaque événement capté
\begin{itemize}
\item Récupérer les clients lié à cet événement. (SOM10)
\item Générer un motif contact (SOM11)
\item Créer le contact prévu à partir de ces données générées (SOM12)
\end{itemize}
\end{itemize}
\end{itemize}

\subsection{Service Objets Metiers}
\subsubsection{SOM1: ConsulterClients }
\begin{itemize}
\item Bloc: Client
\item Rôle: Consulter la liste de clients lié à l'agent donné
\end{itemize}
\subsubsection{SOM2 : RegrouperContacts}
\begin{itemize}
\item Bloc:Contact
\item Entité(s): Contact prévu, Motif contact, Client, Agenda
\item E : idClient
\item S : agenda	
\item Procédure : 	
\begin{itemize}		
\item SB1 : ListeContactPrevus
\item SB2 : ConsulterMotif
\item SB3 : RechercherDatesRDV
\end{itemize}.
\end{itemize}
\subsubsection{SOM3 : AgentAffecte}
\begin{itemize}
\item Bloc: Structure
\item Rôle: Donner l'agent affecté au client donné
\end{itemize}
\subsubsection{SOM4: ConsulterInfosClient}
\begin{itemize}
\item Bloc: Client
\item Rôle: Consulter les informations concernant le client donné
\end{itemize}
\subsubsection{SOM5: ConsulterDatesRDV}
\begin{itemize}
\item Bloc: Agenda
\item Rôle: Consulter les date de rendez-vous d'un contact donné
\end{itemize}
\subsubsection{SM6 : AffecterContactPrevu}
\begin{itemize}
\item Bloc : Contact
\item Entité(s): Contact prévu
\item E : id contact, id agent
\item S : Aucune (L’état du contact passe de prévu à affecté)
\item Procédure : 	
\begin{itemize}		
\item Mettre l'association Concerner la valeur corespondante (idAgent, idContact)
\end{itemize}
\end{itemize}
\subsubsection{SOM7: AnnulerContactPrevu}
\begin{itemize}
\item Bloc : Contact
\item Entité(s): contactPrévu
\item E : id contact, id agent
\item S : Validation d'action
\begin{itemize}		
\item SB: MajEtat(Annulé)[bloc Contact]: passer l'état du contact id en Annulé
\end{itemize}
\end{itemize}
\subsubsection{SOM8: RegrouperContacts}
\begin{itemize}
\item Bloc : Contact
\item Entité(s): Contact prévu
\item E :list $<$listeIdContact $>$\item S : Aucune (L’état du contact passe de prévu à affecté)
\item Procédure : 	
\begin{itemize}		
\item Grouper les contacts ayant comme id passés dans paramètre.
\end{itemize}
\end{itemize}

\subsubsection{SOM9 : recupererDerniersEvenements}
\begin{itemize}
\item Bloc : Evenement
\item Rôle: Donner la liste de événements les plus récemment captés
\end{itemize}

\subsubsection{SOM10 : recupererListClientsLies}
\begin{itemize}
\item Bloc : Evenement
\item Rôle: Donner la liste de clients liés à un événement
\end{itemize}

\subsubsection{SOM11 : creerContactPrevu}
\begin{itemize}
\item Bloc : Contact
\item Rôle: Créer un contact prévu 
\item Entité(s): motif, client, contact
\item E :  événement, liste clients
\item S : contact prévu
\begin{itemize}		
\item SB: genererMotifContact(evenement): Générer le motif contact à partir d'un événement
\item SB: créer contact prévu à partir du motif fourni et de la liste de clients donnée
\end{itemize}
\end{itemize}
